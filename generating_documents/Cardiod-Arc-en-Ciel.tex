%% AMS-LaTeX Created with the Wolfram Language : www.wolfram.com

\documentclass{article}
\usepackage{amsmath, amssymb, graphics, setspace}

\newcommand{\mathsym}[1]{{}}
\newcommand{\unicode}[1]{{}}

\newcounter{mathematicapage}
\begin{document}

\title{Cardiod Arc-en-Ciel}
\author{}
\date{}
\maketitle

June 20, 2020\\
David M. Morgan, Ph.D.\\
Antigonish Landing, NS B2G 2L2 Canada\\
dmmorgan@gmail.com

Algorithm: 

From:\\
\\
http://mathworld.wolfram.com/Cardioid.html\\
\\
The cardioid may also be generated as follows. Draw a circle C and fix a point A on it. Now draw a set of circles centered on the $\quad $circumference of C and passing through A. The envelope of these circles is then a cardioid (Pedoe 1995). 

Code:

Define the circle C and an arbitrary point A:

\begin{doublespace}
\noindent\(\pmb{\text{CircleCPoints} = \text{Table}[}\\
\pmb{\{\text{Cos}[2 \text{Pi} i / 39], \text{Sin}[2 \text{Pi} i / 39]\},}\\
\pmb{\{i, 1, 39\}}\\
\pmb{]}\\
\pmb{}\\
\pmb{\text{PointA} = \{0,1\}}\)
\end{doublespace}

\begin{doublespace}
\noindent\(\left\{\left\{\text{Cos}\left[\frac{2 \pi }{39}\right],\text{Sin}\left[\frac{2 \pi }{39}\right]\right\},\left\{\text{Cos}\left[\frac{4 \pi }{39}\right],\text{Sin}\left[\frac{4 \pi }{39}\right]\right\},\left\{\text{Cos}\left[\frac{2 \pi }{13}\right],\text{Sin}\left[\frac{2 \pi }{13}\right]\right\},\left\{\text{Cos}\left[\frac{8 \pi }{39}\right],\text{Sin}\left[\frac{8 \pi }{39}\right]\right\},\left\{\text{Sin}\left[\frac{19 \pi }{78}\right],\text{Cos}\left[\frac{19 \pi }{78}\right]\right\},\left\{\text{Sin}\left[\frac{5 \pi }{26}\right],\text{Cos}\left[\frac{5 \pi }{26}\right]\right\},\left\{\text{Sin}\left[\frac{11 \pi }{78}\right],\text{Cos}\left[\frac{11 \pi }{78}\right]\right\},\left\{\text{Sin}\left[\frac{7 \pi }{78}\right],\text{Cos}\left[\frac{7 \pi }{78}\right]\right\},\left\{\text{Sin}\left[\frac{\pi }{26}\right],\text{Cos}\left[\frac{\pi }{26}\right]\right\},\left\{-\text{Sin}\left[\frac{\pi }{78}\right],\text{Cos}\left[\frac{\pi }{78}\right]\right\},\left\{-\text{Sin}\left[\frac{5 \pi }{78}\right],\text{Cos}\left[\frac{5 \pi }{78}\right]\right\},\left\{-\text{Sin}\left[\frac{3 \pi }{26}\right],\text{Cos}\left[\frac{3 \pi }{26}\right]\right\},\left\{-\frac{1}{2},\frac{\sqrt{3}}{2}\right\},\left\{-\text{Sin}\left[\frac{17 \pi }{78}\right],\text{Cos}\left[\frac{17 \pi }{78}\right]\right\},\left\{-\text{Cos}\left[\frac{3 \pi }{13}\right],\text{Sin}\left[\frac{3 \pi }{13}\right]\right\},\left\{-\text{Cos}\left[\frac{7 \pi }{39}\right],\text{Sin}\left[\frac{7 \pi }{39}\right]\right\},\left\{-\text{Cos}\left[\frac{5 \pi }{39}\right],\text{Sin}\left[\frac{5 \pi }{39}\right]\right\},\left\{-\text{Cos}\left[\frac{\pi }{13}\right],\text{Sin}\left[\frac{\pi }{13}\right]\right\},\left\{-\text{Cos}\left[\frac{\pi }{39}\right],\text{Sin}\left[\frac{\pi }{39}\right]\right\},\left\{-\text{Cos}\left[\frac{\pi }{39}\right],-\text{Sin}\left[\frac{\pi }{39}\right]\right\},\left\{-\text{Cos}\left[\frac{\pi }{13}\right],-\text{Sin}\left[\frac{\pi }{13}\right]\right\},\left\{-\text{Cos}\left[\frac{5 \pi }{39}\right],-\text{Sin}\left[\frac{5 \pi }{39}\right]\right\},\left\{-\text{Cos}\left[\frac{7 \pi }{39}\right],-\text{Sin}\left[\frac{7 \pi }{39}\right]\right\},\left\{-\text{Cos}\left[\frac{3 \pi }{13}\right],-\text{Sin}\left[\frac{3 \pi }{13}\right]\right\},\left\{-\text{Sin}\left[\frac{17 \pi }{78}\right],-\text{Cos}\left[\frac{17 \pi }{78}\right]\right\},\left\{-\frac{1}{2},-\frac{\sqrt{3}}{2}\right\},\left\{-\text{Sin}\left[\frac{3 \pi }{26}\right],-\text{Cos}\left[\frac{3 \pi }{26}\right]\right\},\left\{-\text{Sin}\left[\frac{5 \pi }{78}\right],-\text{Cos}\left[\frac{5 \pi }{78}\right]\right\},\left\{-\text{Sin}\left[\frac{\pi }{78}\right],-\text{Cos}\left[\frac{\pi }{78}\right]\right\},\left\{\text{Sin}\left[\frac{\pi }{26}\right],-\text{Cos}\left[\frac{\pi }{26}\right]\right\},\left\{\text{Sin}\left[\frac{7 \pi }{78}\right],-\text{Cos}\left[\frac{7 \pi }{78}\right]\right\},\left\{\text{Sin}\left[\frac{11 \pi }{78}\right],-\text{Cos}\left[\frac{11 \pi }{78}\right]\right\},\left\{\text{Sin}\left[\frac{5 \pi }{26}\right],-\text{Cos}\left[\frac{5 \pi }{26}\right]\right\},\left\{\text{Sin}\left[\frac{19 \pi }{78}\right],-\text{Cos}\left[\frac{19 \pi }{78}\right]\right\},\left\{\text{Cos}\left[\frac{8 \pi }{39}\right],-\text{Sin}\left[\frac{8 \pi }{39}\right]\right\},\left\{\text{Cos}\left[\frac{2 \pi }{13}\right],-\text{Sin}\left[\frac{2 \pi }{13}\right]\right\},\left\{\text{Cos}\left[\frac{4 \pi }{39}\right],-\text{Sin}\left[\frac{4 \pi }{39}\right]\right\},\left\{\text{Cos}\left[\frac{2 \pi }{39}\right],-\text{Sin}\left[\frac{2 \pi }{39}\right]\right\},\{1,0\}\right\}\)
\end{doublespace}

\begin{doublespace}
\noindent\(\{0,1\}\)
\end{doublespace}

For each point in CircleCPoints, construct a circle with that point as its centre such that it passes through A; each circle must satisfy:\\
\\
\(r^2\)=\((x-\text{CircleCPoints}[[i,1]])^2+ (y-\text{CircleCPoints}[[i,2]])^2\)\\
\\
in which x and y take on the values of PointA[[1]] and PointA[[2]], respectively, and in which \\
\\
\textit{ r}\\
\\
is the radius of the circle in question. \\
\\
For each point in CircleCPoints, the radius of the required circle is:

\begin{doublespace}
\noindent\(\pmb{\text{ConstructedCircleRadii} = \text{Table}[}\\
\pmb{\text{Sqrt}\left[(\text{PointA}[[1]] - \text{CircleCPoints}[[i,1]])^2 + (\text{PointA}[[2]] - \text{CircleCPoints}[[i,2]])^2\right],}\\
\pmb{\{i, \text{Length}[\text{CircleCPoints}]\}}\\
\pmb{];}\)
\end{doublespace}

Give the third result as an example: 

\begin{doublespace}
\noindent\(\pmb{\text{ConstructedCircleRadii}[[3]]}\)
\end{doublespace}

\begin{doublespace}
\noindent\(\sqrt{\text{Cos}\left[\frac{2 \pi }{13}\right]^2+\left(1-\text{Sin}\left[\frac{2 \pi }{13}\right]\right)^2}\)
\end{doublespace}

Formulate the equation for the upper hemisphere of each circle to be constructed in the form: \\
\\
y = \textit{ f}(x)\\
\\
Recalling that each circle satisfies:\\
\\
\(r^2\)=\((x-\text{CircleCPoints}[[i,1]])^2+ (y-\text{CircleCPoints}[[i,2]])^2\), { } { } (1)\\
\\
one rearranges to solve for y: \\
\\
\(r^2\) - \((x-\text{CircleCPoints}[[i,1]])^2\) = \((y-\text{CircleCPoints}[[i,2]])^2\). { } { } (2)\\
\\
Taking square roots of each side provides:\\
\\
$\unicode{0003}$\(\sqrt{\unicode{0016}r^2-(x-\text{CircleCPoints}[[i,1]])^2 }\) = \((y-\text{CircleCPoints}[[i,2]])\), { } { } (3)\\
\\
and solving for y gives:\\
\\
y = CircleCPoints[[i,2]] + \(\sqrt{\unicode{0016}r^2-(x-\text{CircleCPoints}[[i,1]])^2 }\). { } { }(4)\\
\\
Construct the equations for the upper hemisphere of each circle:

\begin{doublespace}
\noindent\(\pmb{\text{UpperHemisphereEquations} = \text{Table}[}\\
\pmb{\text{CircleCPoints}[[i,2]] + \text{Sqrt}[}\\
\pmb{\text{ConstructedCircleRadii}[[i]]{}^{\wedge}2 - }\\
\pmb{(x - \text{CircleCPoints}[[i,1]]){}^{\wedge}2}\\
\pmb{],}\\
\pmb{\{i, \text{Length}[\text{CircleCPoints}]\}}\\
\pmb{];}\)
\end{doublespace}

Give the third result as an example. 

\begin{doublespace}
\noindent\(\pmb{\text{UpperHemisphereEquations}[[3]]}\)
\end{doublespace}

\begin{doublespace}
\noindent\(\sqrt{-\left(x-\text{Cos}\left[\frac{2 \pi }{13}\right]\right)^2+\text{Cos}\left[\frac{2 \pi }{13}\right]^2+\left(1-\text{Sin}\left[\frac{2 \pi }{13}\right]\right)^2}+\text{Sin}\left[\frac{2 \pi }{13}\right]\)
\end{doublespace}

Plot each upper hemisphere. Note the color formula in PlotStyle. Show the third in the list as an example. 

\begin{doublespace}
\noindent\(\pmb{\text{UpperHemispherePlots} = \text{Table}[}\\
\pmb{\text{Plot}[}\\
\pmb{\text{UpperHemisphereEquations}[[i]],}\\
\pmb{\{}\\
\pmb{x, }\\
\pmb{-\text{ConstructedCircleRadii}[[i]] + \text{CircleCPoints}[[i,1]],}\\
\pmb{\text{ConstructedCircleRadii}[[i]] + \text{CircleCPoints}[[i,1]]}\\
\pmb{\},}\\
\pmb{\text{ColorFunction} \to (\text{ColorData}[\text{{``}Rainbow{''}}][\text{Rescale}[\text{$\#$1},\{-3, 3\}]]\&),}\\
\pmb{\text{ColorFunctionScaling} \to  \text{False},}\\
\pmb{\text{PlotRange} \to  \{\{-3,3\},\{-3,3\}\}}\\
\pmb{],}\\
\pmb{\{i, \text{Length}[\text{ConstructedCircleRadii}]\}}\\
\pmb{];}\)
\end{doublespace}

\begin{doublespace}
\noindent\(\pmb{\text{UpperHemispherePlots}[[3]]}\)
\end{doublespace}

